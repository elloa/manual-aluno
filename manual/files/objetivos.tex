Você é um guerreiro(a)! Venceu um vestibular concorrido, foi aprovado em uma faculdade pública, cursou muitas disciplinas, algumas delas extremamente desafiadoras, e está no passo final para a conclusão do seu curso, a realização do \emph{Trabalho de Conclusão de Curso} (TCC).

O TCC é um dos requisitos parciais para a sua graduação, juntamente com as disciplinas, atividades complementares, etc. É um \emph{trabalho individual} e de \emph{caráter monográfico}, ou seja, de natureza dissertativa que se destina a abordar um assunto em específico. O TCC tem basicamente três objetivos:

\begin{enumerate}
  \item Reunir, aprofundar e sistematizar os conteúdos disponibilizados ao longo das disciplinas do curso em um trabalho de caráter bibliográfico ou prático, relacionado à sua formação;
  \item Concentrar em uma atividade acadêmica as capacidades de criação e de pesquisa do acadêmico no que diz respeito à organização, metodologia, domínio das técnicas de pesquisa, processos de apresentação de trabalho, conhecimentos da pesquisa bibliográfica e da documentação, técnicas de coleta, análise e apresentação de dados, clareza e coerência na redação final;
  \item Contribuir para a criação e disseminação de conhecimento técnico e científico na Computação.
\end{enumerate}

Mas isto não será feito do dia para a noite. Para tanto, você contará com pelo menos três ferramentas essenciais: um orientador; e duas disciplinas (TCC1 e TCC2).

\subsection{Orientador}

\subsection{Disciplinas TCC1 e TCC2}
